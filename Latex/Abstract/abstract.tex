\begin{flushleft}
{\large\textbf{Resumo}}
\end{flushleft}
\noindent
O Large Hadron Collider (LHC) do CERN foi projetado e construído com o objetivo de proporcionar a observação de fenômenos na fronteira das altas energias. O Compact Muon Solenoide (CMS) é um dos quatros experimentos que observam o resultado das colisões hadrônicas de altas energias geradas pelo LHC, podendo medir com grande precisão diversos parâmetros do Modelo Padrão das interações fundamentais bem como propiciar o descobrimento de novos constituintes da matéria e suas interações. Em sua nova fase de operação, iniciada em junho de 2015 e prevista para durar até 2020, ele irá gerar colisões de protons a 13 e 14 TeV, propiciando um ambiente de descobertas em regiões do espaço de fase ainda não acessíveis. O objetivo deste trabalho é investigar a possível existência de ressonâncias pesadas que tenham por sinal característico seu decaimento em bósons vetoriais massivos do Modelo Padrão, em particular em um par de bósons Z. Tais ressonância são previstas de existir em muitas extensões do Modelo Padrão, tais como as que prevem a existência de Dimensões Extras espaciais ou que possuam mecanismos que produzam Matéria Escura. Estudaremos o indício da existência de tais ressonâncias analisando os sinais experimentais surgidos quando um dos bósons Z decai em dois jatos hadrônicos e o outro decai invisivelmente em dois neutrinos.

\begin{flushleft}
{\textbf{Palavras-chave:} Física de Altas Energias; Física de Partículas; Colisores Hadrônicos; Física Além do Modelo Padrão.}
\end{flushleft}

\newpage

\begin{flushleft}
{\large\textbf{Abstract}}
\end{flushleft}
\noindent
The Large Hadron Collider (LHC) at CERN was designed with the purpose to observe new phenomena on the high energies frontier. The Compact Muon Solenoid (CMS) is one of the four experiments that examine the outcome of hadronic collisions at high energies generated by LHC. It can measure with great accuracy various parameters of the Standard Model of fundamental interactions as well as facilitate the discovery of new constituents of matter and their interactions.
In this new stage of operation, initiated in June 2015 and expected to last until 2020, the LHC will generate collisions of protons at energies of up to 13 and 14 TeV, providing an environment of discoveries in regions of phase space still not accessible. The objective of this work is to investigate the existence of heavy resonances which have as characteristic signature, its decay in massive vector bosons of the Standard Model, particularly in a pair of Z bosons.
These kind of resonances are predicted for many extensions of the Standard Model, such as Extra Dimensions or Dark Matter production. We will examine the evidence of such resonances by analyzing the experimental signatures arising when one of the Z bosons decays into two jets and the other decays invisibly into two neutrinos.

\begin{flushleft}
{\textbf{Key-words:} High energy physics; Particle physics; Hadron colliders; Physics beyond standard model.}
\end{flushleft}