\chapter{Introduction}

The Standard Model (SM) of particle physics is the most succesfull theory that we have in order to explain how the basic building blocks of matter interact, governed by four fundamental forces.
Remarkably, the SM provides a profitable description of all current experimental data and represents one of the triumphs of modern physics.\\
\indent
Significant improvement in understanding the SM and its possible extension has been reached in last years, driven in particular by the discovery of the Higgs boson in 2012 \cite{higgsdisco}. The existence of this scalar particle completes the SM, but also provoke many fundamental questions on its properties. 
For example, why the Higgs boson is so much lighter than the Planck mass? One would expect that the large quantum contributions to the square of the Higgs boson mass would inevitably make the mass huge, comparable to the scale at which new physics appears, unless there is an incredible fine-tuning cancellation between the quadratic radiative corrections and the bare mass ("hierarchy problem").\\
\indent
Like this, there are many other questions that the SM can not answer. Given this crossroad, it would seem unavoidable to search for explanation in theories beyond SM (BSM). During the Run2 the LHC will reach energies never before achieved in an accelerator. This opens an unparalleled portal to look for phenomena BSM.\\
\indent
Events comming from a resonance which decay in a pair of vector bosons with an energetic jet and large missing transverse momentum in the final state constitute a clean and distinctive signature in searches for new physics BSM at hadron colliders.\\
\indent
There are several theory models that motivate the existence of resonant massive particles that decay to pairs of bosons. These models intent to explain open questions of the SM such as the incorporation of gravity using extra dimensions. Among the models  are the Randall-Sundrum Warped Extra Dimensions model (RS) \cite{Randall:1999ee,Randall:1999vf} and the Bulk Graviton model \cite{Aga,Fitz}.\\
\indent
This analysis will be based on proton-proton collision data at $\sqrt{s}$ = 13 TeV to be collected by the CMS experiment at the CERN Large Hadron Collider (LHC) during 2015 with an expected luminosity of 1 to 10 fb$^{-1}$. The search strategy focuses on a localised excess in the transverse mass ($M_T$) distribution reconstructed from the transverse momentum of the Z-jet and the missing transverse energy (MET).\\ 
\indent
Although as a starting point we will use the RS model as a Benchmark, this analysis tries to be generic so that wherever possible we will try to perform a model independent search.\\
\indent
The challenge of the analyses described here is the reconstruction of the highly energetic decay products. Since resonances under study have masses of O(TeV), its decay products, i.e. the bosons, have on average transverse momenta of several hundred GeV and above. As a consequence, the particles emerging from the boson decays are very collimated. In particular, the decay products of the hadronically decaying bosons cannot be resolved anymore, but are
instead reconstructed as a single jet object. Dedicated techniques are applied to exploit the substructure of this object.\\
\indent 
Due to the large amount of technical content that involves an analysis of this kind, we will try as far as possible to define the unfamiliar concepts in the best of cases, and in others we choose to simply cite the references.\\
\indent
In chapter I, we give a brief description of the SM of fundamental interactions. In chapter 2 we focus in the LHC accelerator and the CMS detector. In chapter 3, we introduce the physics objects that will be used in this analysis and in
chapter 4 we outline the main steps followed in this analysis, showing preliminary studies done on the selection requirements thresholds in order to reduce the number of background events in comparison with the signal events.

